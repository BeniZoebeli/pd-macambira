\documentclass{article}

\title{pdlua}
\author{Claude Heiland-Allen
\\
\tt{claudiusmaximus@goto10.org}}

\begin{document}

\maketitle

\begin{abstract}
Pd (aka Pure-data) is a real-time visual programming environment
primarily used for multimedia processing. Lua is a powerful, fast,
light-weight, embeddable scripting language. pdlua is a Lua embedding
for Pd.
\end{abstract}

\section{Classes and objects}

\begin{verbatim}
-- example1.pd_lua
local example1 = pd.Class:new():register("example1")
function example1:initialize(sel, atoms)
  return true
end
\end{verbatim}

{\tt pd} is a package automatically available to scripts loaded by pdlua.
pdlua uses a prototype-based object system and {\tt pd.Class} is the
prototype for classes that define patchable objects in Pd.  To create a
new class, use the {\tt :new()} method.  This creates an anonymous class,
which needs to be registered with Pd using the {\tt :register()} method.

The new class {\tt example1} cannot be instantiated yet, as the default
{\tt :initialize()} method returns {\tt false}, indicating to pdlua that
the Pd object should not be created.

With the code above, {\tt example1} can be created in Pd, but it will
have neither inlets nor outlets.

\section{Inlets and methods}

\begin{verbatim}
-- example2.pd_lua
local example2 = pd.Class:new():register("example2")
function example2:initialize(sel, atoms)
  self.inlets = 3
  return true
end
\end{verbatim}

Setting {\tt self.inlets} in the {\tt :initialize()} method will give
the created objects some inlets, in this case three of them.  Not very
interesting yet, as sending messages to these inlets will result in
errors as there are no methods to respond to messages at these inlets.

Messages arriving at the Pd object's inlets are dispatched to the Lua
object's {\tt :in\_*()} methods.  There are five predefined selectors:

\begin{itemize}
\item {\tt bang}
\item {\tt float}
\item {\tt symbol}
\item {\tt pointer}
\item {\tt list}
\end{itemize}

They can be used like this:

\begin{verbatim}
function example2:in_1_bang()
  pd.post("inlet 1 got a bang")
end
function example2:in_1_float(f)
  pd.post("inlet 1 got a float: " .. f)
end
function example2:in_1_symbol(s)
  pd.post("inlet 1 got a symbol: " .. s)
end
function example2:in_1_pointer(p)
  pd.post("inlet 1 got a pointer)
end
function example2:in_1_list(atoms)
  pd.post("inlet 1 got a list: " .. #atoms .. " elements")
end
\end{verbatim}

In the above, the methods are defined for the leftmost inlet.  To add
methods for the other inlets, replace {\tt :in\_1\_*()} with
{\tt :in\_2\_*()} for the second inlet, or {\tt :in\_3\_*()} for the third,
and so on.

It is possible to add methods for other selectors:

\begin{verbatim}
function example2:in_2_baabaa(atoms)
  pd.post("inlet 2 got a baabaa: " .. #atoms .. " elements")
end
function example2:in_2_moomoo(atoms)
  pd.post("inlet 2 got a moomoo: " .. #atoms .. " elements")
end
\end{verbatim}

It is also possible to add methods that catch any selector:

\begin{verbatim}
function example2:in_3(sel, atoms)
  pd.post("inlet 3 got a " .. sel .. ": .. #atoms .. " elements")
end
\end{verbatim}

Or methods that catch messages at any inlet:

\begin{verbatim}
function example2:in_n_float(i, f)
  pd.post("inlet " .. i .. " got a float: " .. f)
end
function example2:in_n_quack(i, atoms)
  pd.post("inlet " .. i .. " got a quack: " .. #atoms .. " elements")
end
\end{verbatim}

Or even catch any message at any inlet:

\begin{verbatim}
function example2:in_n(i, sel, atoms)
  pd.post("inlet " .. i .. " got a " .. sel .. ": " .. #atoms .. " elements")
end
\end{verbatim}

The more specific methods are called before the more general methods:

\begin{itemize}
\item {\tt :in\_1\_selector()}
\item {\tt :in\_n\_selector()}
\item {\tt :in\_1()}
\item {\tt :in\_n()}
\item {\tt :error("no method found")}
\end{itemize}

\section{Outlets}

\section{Sends}

\section{Receives}

\section{Values}

\section{Tables}

\section{Clocks}

\section{Paths}

\end{document}
